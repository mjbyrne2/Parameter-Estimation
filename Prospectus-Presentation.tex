\documentclass{beamer}
\usetheme{Madrid}
\usepackage{amsmath,amsfonts,amssymb}
\usepackage{graphicx,xcolor}
\setbeamertemplate{bibliography item}{\insertbiblabel}

\newcommand{\ToDo}[1]{\textcolor{red}{[#1]}}

% Statistical notation:
\newcommand{\noise}{\eta}	% Noise (single component/variable)
\newcommand{\noiseSD}{\sigma}	% Standard deviation

% Discrete notation:
\newcommand{\gVec}{\mathbf{g}}	% Vector
\newcommand{\gnoise}{\widetilde{g}}	% Component
\newcommand{\gnoiseVec}{\widetilde{\mathbf{g}}}	% Vector
\newcommand{\kVec}{\mathbf{k}}	% Vector
\newcommand{\kMat}{K}	% Matrix K
\newcommand{\fVec}{\mathbf{f}}	% Vector
\newcommand{\tVec}{\mathbf{t}}	% Vector
\newcommand{\dVec}{\mathbf{d}}	% Differential operator vector
\newcommand{\trans}{\mathrm{T}}	% Matrix transpose
\newcommand{\ctrans}{*}	% Conjugate transpose
\DeclareMathOperator{\diag}{diag}	% Diagonal matrix
\DeclareMathOperator{\rank}{rank}	% Rank of a matrix
\newcommand{\noiseVec}{\mathbf{\noise}}	% Noise vector
\newcommand{\singular}{s}	% Singular values
\DeclareMathOperator{\trace}{trace}		% Trace
\newcommand{\dct}[1]{\breve{#1}}	% DCT of vector

% Regularization notation:
\newcommand{\regparam}{\alpha}
\newcommand{\R}{R_{\regparam}}	% Regularization matrix
\newcommand{\regf}{\fVec_{\regparam}}	% Regularized solution
\DeclareMathOperator*{\argmax}{arg\,max}
\DeclareMathOperator*{\argmin}{arg\,min}
\newcommand{\filt}{\phi}
\newcommand{\mfilt}{\psi}

% UPRE derivation notation:
\newcommand{\PE}{\mathbf{p}_{\regparam}}	% Predictive error
\newcommand{\regres}{\mathbf{r}_{\regparam}}	% Regularized residual
\newcommand{\A}{A_{\regparam}}	% Influence matrix
\newcommand{\U}{U}	% UPRE function

% GCV derivation notation:
\newcommand{\GCV}{G}	% GCV function

% Discrepancy principle derivation notation:
\newcommand{\D}{D}	% Discrepancy principle function

\author{Michael Byrne}
\title{Resolution analyses and estimation of regularization parameters for the inversion of integral equations}
\subtitle{Prospectus for Applied Mathematics PhD}
%\setbeamercovered{transparent} 
%\setbeamertemplate{navigation symbols}{} 
%\logo{} 
\institute{Arizona State University} 
\date{December 3, 2019} 
%\subject{} 
\begin{document}

% Title page:
\begin{frame}
\titlepage
\end{frame}

% Table of contents:
\begin{frame}
\tableofcontents
\end{frame}

\section{Introduction}

\begin{frame}
\frametitle{Inverse problems}
\begin{itemize}
\item Inverse problems are concerned with determining model parameters/characteristics from observed data
\item Historical example is the detection of Neptune (1821-1846; see \cite{Airy1847})
\item Modern examples include to deconvolution, tomography, and subsurface mapping from gravitational measurements
\end{itemize}
\begin{block}{Fredholm equation of the first kind}
Given functions $k(x,t)$ and $g(x)$, the Fredholm equation of the first kind is
\[g(x) = \int_a^b k(x,t)f(t)~dt,\]
where the function $f(t)$ is unknown. The function $k$ is called the kernel of the integral equation.
\end{block}
\end{frame}

\begin{frame}
\frametitle{Convolution of functions}
When the kernel has the form $k(x,t) = k(x-t)$ and the limits of integration are infinite, the integral equation represents the convolution of $k$ and $f$:
\[(k * f)(x) := g(x) = \int_{-\infty}^{\infty} k(x-t)f(t)~dt.\]
\begin{itemize}
\item Convolution is a ``smoothing" operation: if $k$ has compact support and $f$ is locally integrable, then $g$ exists and is continuous \cite{DebnathLokenath1999ItHs}
\item \ToDo{Insert a convolution animation here}
\end{itemize}
\end{frame}

\begin{frame}
\frametitle{Continuous to discrete setting (1D)}
In many applications, data are finite (i.e. samples of functions)
\[\gVec = \begin{bmatrix}
g_0 \\
\vdots \\
g_{n-1}
\end{bmatrix}, \quad g_j = \sum_{\ell=-\infty}^{\infty} k_{j-\ell}f_{\ell}\]
\[(\ldots,f_{-m+1},\ldots,f_0,f_1,\ldots,f_{n-1},f_n,\ldots,f_{n+m},\ldots)\]
\[(\ldots,0,0,k_{-m+1},k_{-m+2},\ldots,k_0,\ldots,k_{m-2},k_{m-1},0,0,\ldots)\]
\ToDo{be able to explain this i.e. weighting, existence ($\ell^1$), infinite vs finite matrix, discrete convolution} \\
\ToDo{talk about the indices in relation to sample points (collocation etc.)}
\ToDo{Insert two pictures: one of sampled $f$ and one for sampled $k$}
\end{frame}

\begin{frame}
\frametitle{Continuous to discrete setting (1D)}
Assuming the kernel $k$ has compact support, the resulting discrete system is finite and $\gVec$ can be expressed as the following matrix product:
\[\resizebox{.9 \textwidth}{!}{$
\begin{bmatrix}
k_{m-1} & \cdots & k_0 & \cdots & k_{-m+1} & & & & & \\
 & k_{m-1} & & k_0 & & k_{-m+1} & & & 0 & \\
 & & \ddots & \ddots & \ddots & \ddots & \ddots & & & \\
 & & & \ddots & \ddots & \ddots & \ddots & \ddots & & \\
 & 0 & & & k_{m-1} & & k_0 & & k_{-m+1} & \\
 & & & & & k_{m-1} & \cdots & k_0 & \cdots & k_{-m+1}
\end{bmatrix}\begin{bmatrix}
f_{-m} \\
\vdots \\
f_{-1} \\
f_0 \\
\vdots \\
f_{n-1} \\
f_n \\
\vdots \\
f_{n+m-1}
\end{bmatrix}$}\]
However, this system is underdetermined without imposing some boundary conditions on $f$.
\end{frame}

\begin{frame}
\frametitle{Singular value decomposition (SVD)}
Given a system $\gnoiseVec = \kMat\fVec + \noiseVec$ where $\kMat$ is an $m \times n$ matrix and $\rank(\kMat) = r$, direct matrix inversion is impractical/impossible. A theoretical alternative is to express $\kMat$ using the (compact) SVD.
\[\kMat = U\Sigma{V^\ctrans}\]
\begin{itemize}
\item $U$ is $m \times r$, $V^\ctrans$ is $r \times n$, and both $U$ and $V$ are column orthonormal.
\item $\Sigma = \diag(\singular_0,\ldots,\singular_{r-1})$ where $s_0 \geq \ldots \geq \singular_{r-1} > 0$ (the singular values)
\item If $\kMat$ is singular, the pseudoinverse $K^\dagger = V\Sigma^\dagger{U}^\ctrans$ with $\Sigma^\dagger = \diag\left(\frac{1}{\singular_0},\ldots,\frac{1}{\singular_{r-1}}\right)$ can be used for obtain a solution:
\[\kMat^\dagger{\gVec} = (V\Sigma^\dagger{U}^\ctrans)(\kMat\fVec + \noiseVec) = \fVec + \sum_{\ell=0}^{r-1} \frac{(U_{\cdot,\ell})^\ctrans{\noiseVec}}{\singular_{\ell}}V_{\cdot,\ell}\]
\end{itemize}
However, division by small singular values results in numerical instability. \\
\ToDo{The noise vector should be bold}
\end{frame}

\begin{frame}
\frametitle{Regularization}
\[\fVec_\regparam = \sum_{\ell=0}^{r-1} \filt(\regparam,\singular_\ell)\frac{(U_{\cdot,\ell})^\ctrans{\noiseVec}}{\singular_{\ell}}V_{\cdot,\ell}\]
To prevent numerical instability, the summands can be multiplied by \textit{filter functions} $\filt(\regparam,\singular_{\ell})$, dependent upon a non-negative \textit{regularization parameter} $\regparam$.
\begin{itemize}
\item Desirable to have $\filt(\regparam,\singular_\ell)/\singular_\ell \approx 1$ for large $\singular_\ell$ and $\filt(\regparam,\singular_\ell)/\singular_\ell \approx 0$ for small $\singular_\ell$
\item Simple example is 
\[\filt(\regparam,\singular_\ell) = \begin{cases}
1, & \singular^2_\ell > \regparam \\
0, & \singular^2_\ell \leq \regparam
\end{cases},\]
which corresponds to the truncated singular value decomposition (TSVD) \cite{Vogel:2002}
\end{itemize}
\end{frame}

\begin{frame}
\frametitle{Tikhonov regularization}
Choosing the following filter function corresponds to \textit{Tikhonov regularization} \cite{Tikh1963}
\[\filt(\regparam,\singular_\ell) = \frac{\singular^2_\ell}{\singular^2_\ell + \regparam}\]
Tikhonov regularization can stated as a minimization problem:
\begin{block}{Tikhonov regularization}
The solution obtained by Tikhonov regularization is
\[\fVec_\regparam = \argmin_{\fVec\in\mathbb{R}^n} \left\{\|\kMat\fVec - \gnoiseVec\|^2 + \regparam^2\|D\fVec\|^2\right\},\]
where $D$ is the matrix representation of a linear operator and $\|\cdot\|$ is the 2-norm. The term $\|D\fVec\|^2$ is an example of a \textit{penalty function} \cite{Vogel:2002}. In matrix form, $\regf = (\kMat^\ctrans{\kMat} + \regparam{D^\ctrans}D)^{-1}\kMat^\ctrans\gnoiseVec$.
\end{block}
Using the generalized SVD, the problem can be recast so that $D$ is ``absorbed" into the matrix $\kMat$ \ToDo{cite ABT chapter and Hansen}
\end{frame}

\begin{frame}
\frametitle{Goals and overview of remaining presentation}
\textcolor{red}{The primary goal is to determine the effects of downsampling on the efficacy of regularization parameters selected using a variety of methods.} \\
\begin{itemize}
\item Continue transition from continuous to discrete setting (1D)
\item Methodology of downsampling
\item Parameter selection methods
\begin{itemize}
\item Unbiased predictive risk estimator
\item Generalized cross validation
\item Discrepancy principle
\end{itemize}
\item DCT and DFT approach
\begin{itemize}
\item Discussion of boundary conditions
\item Discrete system/matrix structure
\item Statistical results and numerical examples
\end{itemize}
\item Machine learning approach
\item Conclusion and future work
\end{itemize}
\end{frame}

\begin{frame}
\frametitle{Continuous to discrete setting (1D)}
The discrete system can be rewritten as
\[\gVec = T_{l}\fVec_{l} + T\fVec + T_{r}\fVec_{r}\]
where matrices/vectors come from the following partition \cite{NeumannDCT}:
\[\resizebox{1 \textwidth}{!}{$
\left[\begin{array}{cc|cccccc|cc}
k_{m-1} & \cdots & k_0 & \cdots & k_{-m+1} & & & & & \\
 & k_{m-1} & & k_0 & & k_{-m+1} & & & 0 & \\
 & & \ddots & \ddots & \ddots & \ddots & \ddots & & & \\
 & & & \ddots & \ddots & \ddots & \ddots & \ddots & & \\
 & 0 & & & k_{m-1} & & k_0 & & k_{-m+1} & \\
 & & & & & k_{m-1} & \cdots & k_0 & \cdots & k_{-m+1} 
\end{array}\right]\left[\begin{array}{c}
f_{-m} \\
\vdots \\
f_{-1} \\
\hline 
f_0 \\
\vdots \\
f_{n-1} \\
\hline
f_n \\
\vdots \\
f_{n+m-1}
\end{array}\right]\begin{array}{c}
\vspace*{0.25 in} \\
\leftarrow \fVec_l \\
\vspace*{0.25 in} \\
\leftarrow \fVec \\
\vspace*{0.25 in} \\
\leftarrow \fVec_r \\
\vspace*{0.25 in} 
\end{array}$}\]
\ToDo{Expand on the shape of $T_l$ and $T_r$ on the whiteboard}
\end{frame}

\begin{frame}
\frametitle{Downsampling}
For two natural numbers $B$ and $N$, let $n = B^N$. The map $\downarrow_B:\mathbb{C}^n \rightarrow \mathbb{C}^{n/B}$ is defined by $\mathbf{y} = \downarrow_B(\mathbf{x})$, where $\mathbf{y}$ is the vector formed by concatenating every $B$th element of $\mathbf{x}$ \cite{AudioDFT}.
\begin{itemize}
\item For example, $\downarrow_2([1,2,3,4,5,6]) = [1,3,5]$
\item $B$-downsampling has a matrix representation
\item $B = 2$ is advantageous for FFT-type algorithms \ToDo{cite this}
\end{itemize}
\ToDo{including animation of this process for $B = 2$}
\end{frame}

\begin{frame}
\frametitle{Numerical examples}
Assuming $g$ is sampled over $[0,1]$ and the support of $k$ is $[-\frac{1}{2},\frac{1}{2}]$, $f$ must be defined on $[-\frac{1}{2},\frac{3}{2}]$:
\[g(x) = \int_{-\infty}^{\infty} k(x-t)f(t)~dt = \int_{x+\frac{1}{2}}^{x-\frac{1}{2}} k(x-t)f(t)~dt\]
\ToDo{be able to explain this} \\
Further assume that $g_j = g(x_j)$ with $x_j = jh$ for $h = 1/n$ and $j = 0,\ldots,n-1$. Extending the collocation points to serve as quadrature points in $t$ provides the setup for the discrete linear system. Under the above assumptions (\ToDo{$n$ even}), $m = n/2$ and the quadrature points are $t_j = jh$ for $j = -m+1,\ldots,n+m-1$
\[g_j \approx \sum_{k=-m+1}^{n+m-2} k_{j-k}f_{k} = \sum_{\ell=m-1}^{m+1} k_{\ell}f_{j-\ell}, \quad j = 0,\ldots,n-1\]
\end{frame}

\section{Parameter estimation methods}

\begin{frame}
\frametitle{Unbiased predictive risk estimator (UPRE)}
The UPRE method \cite{Mallows1973} is to choose $\regparam$ as a minimizer of the function
\[\U(\regparam) = \frac{1}{n}\|\regres\|^2 + \frac{2\noiseSD^2}{n}\trace(\A) - \noiseSD^2,\]
where $\A = \kMat(\kMat^\ctrans{\kMat} + \regparam{D^\ctrans}D)^{-1}\kMat^\ctrans$ is the Tikhonov influence matrix and $\regres = (\A - I)\gnoiseVec$ is the \textit{regularized residual}.
\begin{itemize}
\item Requires knowledge of the noise variance $\noiseSD^2$
\item $\U(\regparam)$ can have more than one minimizer
\end{itemize}
\end{frame}

\begin{frame}
\frametitle{Generalized cross validation (GCV)}
The GCV method \cite{Wahba1977,Wahba1990} is to choose $\regparam$ as a minimizer of the function
\[\GCV(\regparam) = \frac{\dfrac{1}{n}\|\regres\|^2}{\left[\dfrac{1}{n}\trace(I-\A)\right]^2} = \frac{n\|\regres\|^2}{\left[\trace(I-\A)\right]^2},\]
where $\A$ is the Tikhonov influence matrix and $\regres$ is the regularized residual.
\begin{itemize}
\item In contrast to the UPRE method, the GCV method does not require knowledge of $\noiseSD^2$
\end{itemize}
\end{frame}

\begin{frame}
\frametitle{The discrepancy principle}
The discrepancy principle \cite{Morozov1966} says to choose $\regparam$ as a root of the function
\[\D(\regparam) = \frac{1}{n}\|\regres\|^2 - \noiseSD^2\]
where $\regres$ is the regularized residual.
\begin{itemize}
\item Similar to the UPRE method, applying the discrepancy principle requires knowledge of $\noiseSD^2$
\item In contrast* to the UPRE and GCV methods, the discrepancy principle relies on finding a root of a function instead of finding a minimizer
\item Unlike $\U(\regparam)$ and $\GCV(\regparam)$, $\D(\regparam)$ is monotone increasing for all $\regparam > 0$
\end{itemize}
\end{frame}

\section{DCT approach}

\begin{frame}
\frametitle{Neumann boundary condition}
\ToDo{Insert visual representation of what the condition does to a graph} \\
Letting $J$ be the reversal matrix, the system $\gVec = T_{l}\fVec_{l} + T\fVec + T_{r}\fVec_{r}$ becomes
\[\gVec = [(\mathbf{0}~|~T_{l})J + T + (T_{r}~|~\mathbf{0})J]\fVec = A\fVec,\]
where $\mathbf{0}$ is a zero matrix and $A$ is a Toeplitz-plus-Hankel matrix.
\begin{itemize}
\item When the kernel $\kVec$ is symmetric (meaning that $k_j = k_{-j}$ for all $j \in \{-m+1,\ldots,m-1\}$), Toeplitz-plus-Hankel matrices are diagonalized by the DCT \cite{Martucci1994,NeumannDCT}
\item The DCT only requires real operation and is thus about twice as fast as using the DFT \cite{RaoYip2014} \ToDo{find page range}
\item For 2D problems, the corresponding system matrices are block Toeplitz-plus-Hankel with Toeplitz-plus-Hankel blocks
\end{itemize}
\end{frame}

\begin{frame}
\frametitle{Definition of the DCT}
There are eight version of the DCT, each corresponding to how the left and right endpoints of an interval are handled.
\begin{block}{DCT-II}
Given $\mathbf{x}$, the DCT-II of $\mathbf{x}$, denoted $\dct{\mathbf{x}}$, is defined as
\[\dct{x}_j = \sqrt{\frac{2 - \delta_{j,0}}{N}} \sum_{k=0}^{N-1} \cos\left(\frac{\pi{j}(2k + 1)}{2N}\right) x_k.\]
\end{block}
\end{frame}

\begin{frame}
\frametitle{DCT-version of UPRE}

\end{frame}

\begin{frame}
\frametitle{DCT-version of GCV}

\end{frame}

\begin{frame}
\frametitle{DCT-version of the discrepancy principle}

\end{frame}

\section{DFT approach}

\begin{frame}
\frametitle{Periodic boundary condition}
\ToDo{Insert visual representation of what the condition does to a graph} \\
Letting $J$ be the reversal matrix, the system $\gVec = T_{l}\fVec_{l} + T\fVec + T_{r}\fVec_{r}$ becomes
\[\gVec = [(\mathbf{0}~|~T_{l}) + T + (T_{r}~|~\mathbf{0})]\fVec = C\fVec,\]
where $\mathbf{0}$ is a zero matrix and $C$ is a circulant matrix.
\begin{itemize}
\item Circulant matrices are diagonalized by the DFT
\item Due to the fast Fourier transform (FFT), the cost of using the DFT is $O(n\log(n))$ instead of $O(n^2)$ \cite{CooleyTukey}
\item For 2D problems, the corresponding system matrices are block circulant with circulant blocks
\end{itemize}
\end{frame}

\begin{frame}
\frametitle{Definition of the DFT}

\end{frame}

\begin{frame}
\frametitle{DFT-version of UPRE}
If $D = F^\ctrans\Lambda{F}$ \ToDo{have an example of an operator that is diagonalized by the DFT} with $\Lambda = \diag(\widehat{\dVec})$, then
\[\frac{1}{n}\|\regres\|^2 = \frac{1}{n}\|F\regres\|^2 = \frac{1}{n}\|F(F^\ctrans\Delta{F})\regf - F\gnoiseVec\|^2 = \frac{1}{n}\|\Delta\widehat{\regf} - \widehat{\gnoiseVec}\|^2.\]
Thus the product $\Delta\widehat{\regf}$ can be written as
\[\Delta\widehat{\regf} = \Delta{F}[(C^\ctrans{C} + \regparam{D^\ctrans}D)^{-1}C^\ctrans\gnoiseVec] = \Delta(\Delta^\ctrans{\Delta} + \regparam\Lambda^\ctrans{\Lambda})^{-1}\Delta^\ctrans\widehat{\gnoiseVec}.\]
Therefore
\begin{align*}
\frac{1}{n}\|\regres\|^2 &= \frac{1}{n}\|\Delta(\Delta^\ctrans{\Delta} + \regparam\Lambda^\ctrans{\Lambda})^{-1}\Delta^\ctrans\widehat{\gnoiseVec} - \widehat{\gnoiseVec}\|^2 \\
&= \frac{1}{n}\|(\Delta(\Delta^\ctrans{\Delta} + \regparam\Lambda^\ctrans{\Lambda})^{-1}\Delta^\ctrans - I)\widehat{\gnoiseVec}\|^2.
\end{align*}
If $D = I$, then
\[\frac{1}{n}\|\regres\|^2 = \frac{1}{n}\sum_{j=0}^{n-1}|\widehat{\gnoise}_j|^2\left(\mfilt(\regparam,|\widehat{k}_j|^2)\right)^2.\]
\end{frame}

\begin{frame}
\frametitle{DFT-version of UPRE}
The Tikhonov influence matrix $\A$ can be expressed as
\[\A = C(C^\ctrans{C} + \regparam{D^\ctrans}D)^{-1}C^\ctrans = F^\ctrans\Delta(\Delta^\ctrans{D} + \regparam\Lambda^\ctrans{\Lambda})^{-1}\Delta^\ctrans{F}.\]
Since $\Delta(\Delta^\ctrans{D} + \regparam\Lambda^\ctrans{\Lambda})^{-1}\Delta^\ctrans$ is diagonal,
\[\trace(\A) = \sum_{j = 0}^{n-1} \frac{|\widehat{k}_j|^2}{|\widehat{k}_j|^2 + \regparam|\widehat{d}_j|^2} = \sum_{j = 0}^{n-1} \filt(\regparam,|\widehat{k}_j|),\]
where the last equality is obtained when $D = I$.
\begin{block}{DFT-version of the UPRE function}
Under the assumptions that $C$ is circulant and $D = I$,
\[\U(\regparam) = .\]
\end{block}
\end{frame}

\begin{frame}
\frametitle{DFT-version of GCV}

\end{frame}

\begin{frame}
\frametitle{DFT-version of the discrepancy principle}

\end{frame}

\section{Machine learning approach}

\begin{frame}
\frametitle{Machine learning approach}

\end{frame}

\section{Conclusion}

\begin{frame}
\frametitle{Summary of presentation}

\end{frame}

\begin{frame}
\frametitle{Future work}
3D gravity problems
\end{frame}

\begin{frame}[allowframebreaks]
\frametitle{References}
\bibliographystyle{siam}
\bibliography{Parameter-Estimation}
\end{frame}

\end{document}